% Options for packages loaded elsewhere
\PassOptionsToPackage{unicode}{hyperref}
\PassOptionsToPackage{hyphens}{url}
%
\documentclass[
  ignorenonframetext,
]{beamer}
\usepackage{pgfpages}
\setbeamertemplate{caption}[numbered]
\setbeamertemplate{caption label separator}{: }
\setbeamercolor{caption name}{fg=normal text.fg}
\beamertemplatenavigationsymbolsempty
% Prevent slide breaks in the middle of a paragraph
\widowpenalties 1 10000
\raggedbottom
\setbeamertemplate{part page}{
  \centering
  \begin{beamercolorbox}[sep=16pt,center]{part title}
    \usebeamerfont{part title}\insertpart\par
  \end{beamercolorbox}
}
\setbeamertemplate{section page}{
  \centering
  \begin{beamercolorbox}[sep=12pt,center]{part title}
    \usebeamerfont{section title}\insertsection\par
  \end{beamercolorbox}
}
\setbeamertemplate{subsection page}{
  \centering
  \begin{beamercolorbox}[sep=8pt,center]{part title}
    \usebeamerfont{subsection title}\insertsubsection\par
  \end{beamercolorbox}
}
\AtBeginPart{
  \frame{\partpage}
}
\AtBeginSection{
  \ifbibliography
  \else
    \frame{\sectionpage}
  \fi
}
\AtBeginSubsection{
  \frame{\subsectionpage}
}

\usepackage{amsmath,amssymb}
\usepackage{iftex}
\ifPDFTeX
  \usepackage[T1]{fontenc}
  \usepackage[utf8]{inputenc}
  \usepackage{textcomp} % provide euro and other symbols
\else % if luatex or xetex
  \usepackage{unicode-math}
  \defaultfontfeatures{Scale=MatchLowercase}
  \defaultfontfeatures[\rmfamily]{Ligatures=TeX,Scale=1}
\fi
\usepackage{lmodern}
\ifPDFTeX\else  
    % xetex/luatex font selection
\fi
% Use upquote if available, for straight quotes in verbatim environments
\IfFileExists{upquote.sty}{\usepackage{upquote}}{}
\IfFileExists{microtype.sty}{% use microtype if available
  \usepackage[]{microtype}
  \UseMicrotypeSet[protrusion]{basicmath} % disable protrusion for tt fonts
}{}
\makeatletter
\@ifundefined{KOMAClassName}{% if non-KOMA class
  \IfFileExists{parskip.sty}{%
    \usepackage{parskip}
  }{% else
    \setlength{\parindent}{0pt}
    \setlength{\parskip}{6pt plus 2pt minus 1pt}}
}{% if KOMA class
  \KOMAoptions{parskip=half}}
\makeatother
\usepackage{xcolor}
\newif\ifbibliography
\setlength{\emergencystretch}{3em} % prevent overfull lines
\setcounter{secnumdepth}{-\maxdimen} % remove section numbering


\providecommand{\tightlist}{%
  \setlength{\itemsep}{0pt}\setlength{\parskip}{0pt}}\usepackage{longtable,booktabs,array}
\usepackage{calc} % for calculating minipage widths
\usepackage{caption}
% Make caption package work with longtable
\makeatletter
\def\fnum@table{\tablename~\thetable}
\makeatother
\usepackage{graphicx}
\makeatletter
\def\maxwidth{\ifdim\Gin@nat@width>\linewidth\linewidth\else\Gin@nat@width\fi}
\def\maxheight{\ifdim\Gin@nat@height>\textheight\textheight\else\Gin@nat@height\fi}
\makeatother
% Scale images if necessary, so that they will not overflow the page
% margins by default, and it is still possible to overwrite the defaults
% using explicit options in \includegraphics[width, height, ...]{}
\setkeys{Gin}{width=\maxwidth,height=\maxheight,keepaspectratio}
% Set default figure placement to htbp
\makeatletter
\def\fps@figure{htbp}
\makeatother

\makeatletter
\@ifpackageloaded{caption}{}{\usepackage{caption}}
\AtBeginDocument{%
\ifdefined\contentsname
  \renewcommand*\contentsname{Table of contents}
\else
  \newcommand\contentsname{Table of contents}
\fi
\ifdefined\listfigurename
  \renewcommand*\listfigurename{List of Figures}
\else
  \newcommand\listfigurename{List of Figures}
\fi
\ifdefined\listtablename
  \renewcommand*\listtablename{List of Tables}
\else
  \newcommand\listtablename{List of Tables}
\fi
\ifdefined\figurename
  \renewcommand*\figurename{Figure}
\else
  \newcommand\figurename{Figure}
\fi
\ifdefined\tablename
  \renewcommand*\tablename{Table}
\else
  \newcommand\tablename{Table}
\fi
}
\@ifpackageloaded{float}{}{\usepackage{float}}
\floatstyle{ruled}
\@ifundefined{c@chapter}{\newfloat{codelisting}{h}{lop}}{\newfloat{codelisting}{h}{lop}[chapter]}
\floatname{codelisting}{Listing}
\newcommand*\listoflistings{\listof{codelisting}{List of Listings}}
\makeatother
\makeatletter
\makeatother
\makeatletter
\@ifpackageloaded{caption}{}{\usepackage{caption}}
\@ifpackageloaded{subcaption}{}{\usepackage{subcaption}}
\makeatother
\ifLuaTeX
  \usepackage{selnolig}  % disable illegal ligatures
\fi
\usepackage{bookmark}

\IfFileExists{xurl.sty}{\usepackage{xurl}}{} % add URL line breaks if available
\urlstyle{same} % disable monospaced font for URLs
\hypersetup{
  pdftitle={Small Open Economy Extension (IRBC)},
  pdfauthor={Pablo Winant},
  hidelinks,
  pdfcreator={LaTeX via pandoc}}

\title{Small Open Economy Extension (IRBC)}
\subtitle{Macro II - Fluctuations}
\author{Pablo Winant}
\date{}

\begin{document}
\frame{\titlepage}

\section{Introduction and Basic
Facts}\label{introduction-and-basic-facts}

\begin{frame}{Why a small open economy?}
\phantomsection\label{why-a-small-open-economy}
What are the classical reasons to open economy to trade

\begin{itemize}[<+->]
\tightlist
\item
  trade integration

  \begin{itemize}[<+->]
  \tightlist
  \item
    taste for variety
  \item
    comparative advantage
  \end{itemize}
\item
  financial integration

  \begin{itemize}[<+->]
  \tightlist
  \item
    smooth shock / insurance
  \end{itemize}
\end{itemize}
\end{frame}

\begin{frame}{From RBC to IRBC}
\phantomsection\label{from-rbc-to-irbc}
After the success of RBC models to match business cycles it didn't take
long before the same methodology was applied to International Business
Cycles

\begin{itemize}
\tightlist
\item
  Most famous: Backus, Kehoe, Kydland (1992) (freshwater economists)
\end{itemize}

Very successful methodology:

\begin{itemize}
\tightlist
\item
  facts at odd with theoretical predictions have been called ``puzzles''
\end{itemize}
\end{frame}

\begin{frame}{Moments}
\phantomsection\label{moments}
\begin{figure}[H]

{\centering \includegraphics{assets/moments.png}

}

\caption{Business Cycles}

\end{figure}%

From Kehoe,Kydland (1995)
\end{frame}

\begin{frame}{Comovements}
\phantomsection\label{comovements}
\includegraphics{assets/comovements.png}
\end{frame}

\begin{frame}{Stylized Facts}
\phantomsection\label{stylized-facts}
\begin{itemize}[<+->]
\tightlist
\item
  Domestically:

  \begin{itemize}[<+->]
  \tightlist
  \item
    output more variable than consumption
  \item
    output highly correlated
  \item
    productivity strongly procyclical
  \item
    trade balance strongly countercyclcal
  \item
    positive comovements in output
  \end{itemize}
\item
  Internationally:

  \begin{itemize}[<+->]
  \tightlist
  \item
    smaller comovements in consumption

    \begin{itemize}[<+->]
    \tightlist
    \item
      Backus-Kehoe-Kydland puzzle
    \end{itemize}
  \end{itemize}
\end{itemize}
\end{frame}

\section{Modeling a Small Open
Econmomy}\label{modeling-a-small-open-econmomy}

\begin{frame}{Endowment model}
\phantomsection\label{endowment-model}
Take an endowment economy: \((y_t)_t\) is exogenously given.

\[\max_{c_t} \sum_{t=0}^{\infty} \beta^t u(c_t)\]

\[c_t+a_{t+1} \leq y_t + (1+r) a_t\]

Country takes as given world interest rate \(r\).

\begin{itemize}
\tightlist
\item
  a small open economy doesn't affect world prices
\end{itemize}
\end{frame}

\begin{frame}{Endowment model (2)}
\phantomsection\label{endowment-model-2}
We solve this problem with the terminal conditions:

\begin{itemize}
\item
  \(a_0\) given
\item
  \(\lim_{T\rightarrow\infty} \frac{a_{T+1}}{(1+r)^T}\geq0\)

  \begin{itemize}
  \tightlist
  \item
    \emph{no-ponzi} condition
  \end{itemize}
\end{itemize}
\end{frame}

\begin{frame}{Endowment Model (3)}
\phantomsection\label{endowment-model-3}
First order conditions:

\begin{align}
u^{\prime}(c_t)& =& \lambda_t \\
 \lambda_t &=& \beta (1+r) \lambda_{t+1}
\end{align}

Under the \emph{technical assumption} \(\beta (1+r)=1\) we get:

\[c_0 = \frac{r}{1+r}\left\{ (1+r) a_0 + \sum_{t=0}^{\infty} \frac{y_t}{(1+r)^t}\right\}\]

\pause

\begin{itemize}
\tightlist
\item
  problem isomorphic to consumption-savings decisions
\item
  consumption is determined by permanent income
\end{itemize}
\end{frame}

\begin{frame}{Intertemporal approach to the current account}
\phantomsection\label{intertemporal-approach-to-the-current-account}
Observe the formula:
\[c_0 = \frac{r}{1+r}\left\{ (1+r) a_0 + \sum_{t=0}^{\infty} \frac{y_t}{(1+r)^t}\right\}\]

What does it tell us about the current account?

\pause

\begin{itemize}
\tightlist
\item
  trade balance responds positively to temporary shock in \(y_t\)
\item
  and to news about future income shocks:

  \begin{itemize}
  \tightlist
  \item
    This is the \emph{intertemporal approach} to the current account
  \end{itemize}
\end{itemize}
\end{frame}

\begin{frame}{Unit root}
\phantomsection\label{unit-root}
Still with the same formula:
\[c_0 = \frac{r}{1+r}\left\{ (1+r) a_0 + \sum_{t=0}^{\infty} \frac{y_t}{(1+r)^t}\right\}\]

What is the effect of an increase in \(a_0\)?

\begin{itemize}
\tightlist
\item
  consumption rises permanently

  \begin{itemize}
  \tightlist
  \item
    by small amount \(r\) corresponding to interests
  \end{itemize}
\item
  there is a unit root
\end{itemize}
\end{frame}

\begin{frame}{Exercise}
\phantomsection\label{exercise}
From the first order conditions

\begin{align}
u^{\prime}(c_t) & = & \lambda_t \\
\lambda_t & = &  \beta (1+r) \lambda_{t+1}
\end{align}

assuming \(u(c_t) = \log (c_t)\), can you get the equation for the law
of motion of \(a_t\) and show the presence of a unit root?
\end{frame}

\begin{frame}{Adding capital}
\phantomsection\label{adding-capital}
We add capital and production to our endowment economy:
\[y_t = z_t k_t^\alpha\] \[k_t = (1-\delta) k_{t-1} + i_{t-1}\]

The aggregate resource constraint:

\[a_{t+1} + c_t + i_t = (1+r) a_t + z_t\]
\end{frame}

\begin{frame}{Adding capital: optimality conditions}
\phantomsection\label{adding-capital-optimality-conditions}
We get first order conditions

\[\lambda_t = \beta \lambda_{t+1} (1+r)\]
\[\lambda_t = \beta \lambda_{t+1}\left[ (1-\delta) + z_{t+1} f^{\prime}(k_{t+1}) \right]\]

where \(\lambda_t\) is lagrange multiplier associated to budget
constraint.
\end{frame}

\begin{frame}{Adding capital: first order conditions}
\phantomsection\label{adding-capital-first-order-conditions}
Since \(\lambda_t\) (constraint is always binding), we get:
\[k_{t+1} = \left( \frac{r+\delta}{a z_{t+1}}\right)^{\frac{1}{\alpha-1}}\]

and investment
\[i_t = \left( \frac{r+\delta}{a z_{t+1}}\right)^{\frac{1}{\alpha-1}}- (1-\delta)\left( \frac{r+\delta}{a z_{t}}\right)^{\frac{1}{\alpha-1}}\]

Since investment comoves one to one with productivity shock
(conditionnally on \(z_{t-1}\)), it is way too volatile.
\end{frame}

\begin{frame}{Add friction to the investment}
\phantomsection\label{add-friction-to-the-investment}
A possible solution: change the resource constraint such that adjusting
capital is costly

\[a_{t+1} + c_t + i_t + \frac{\omega}{2}\frac{(i_t-\delta k^{\star})}{k_t} = (1+r)a_t + z f(k_t)\]

\[k_{t+1} = (1-\delta) k_t + i_t\]

where \(\omega\) is an adjustment friction. Typically, \(\omega\) is
chosen so that the model replicates \(\frac{Var(i_t)}{Var(y_t)}\) from
the data.
\end{frame}

\begin{frame}{How to make the distribution stationary?}
\phantomsection\label{how-to-make-the-distribution-stationary}
Unit root (see time series VAR for more details)
\[a_t = a_{t-1} + ... \text{other variables in t-1} + \text{shocks in t}\]

Problem:

\begin{itemize}
\tightlist
\item
  there isn't a unique deterministic steady-state
\item
  the ergodic distribution of the model variables is not defined
\end{itemize}

This raises practical issues (notably for estimation) for the
\emph{linear} model.
\end{frame}

\begin{frame}{How to get rid of the unit root?}
\phantomsection\label{how-to-get-rid-of-the-unit-root}
General idea: introduce a force that pulls the level of foreign assets
towards equilibrium

Schmitt Grohe and Uribe (2004) consider many options:

\begin{itemize}
\tightlist
\item
  debt-elastic interest rate: \[1+r = 1+r^{\star} + \pi(a_d)\]

  \begin{itemize}
  \tightlist
  \item
    with \(\pi(0)=0\) and \(\pi^{\prime}(0)>0\)
  \item
    \(\pi\) can be understood as a risk premium on rising debt
  \end{itemize}
\end{itemize}

\begin{itemize}
\tightlist
\item
  endogenous time-discount (aka Usawa preferences)
  \[\beta(c_t) = (1+c_t)^{-\chi}\]
\item
  costs of adjustment for international portfolios
\end{itemize}

\pause

SGU show that the choice of the stationarization device has little
effect for the dynamics (moments) of most variables
\end{frame}

\section{A benchmark Small Open Economy
Model}\label{a-benchmark-small-open-economy-model}

\begin{frame}{A benchmark Small Open Economy Model}
\phantomsection\label{a-benchmark-small-open-economy-model-1}
\emph{Closing Small Economy Models}, Schmitt Grohe and Uribe (2003), JIE

\begin{itemize}
\tightlist
\item
  small open economy model with production, consumption-leisure tradeoff
  and capital adjustment costs
\item
  RBC+open+adj costs
\end{itemize}
\end{frame}

\begin{frame}{The model}
\phantomsection\label{the-model}
\[\max_{c_t, n_t} \sum_{t=0}^{\infty} \beta^t u(c_t)\]

\[c_t + k_{t+1} + a_{t+1} = y_t + g_t - \frac{\omega}{2}(k_{t+1}-k_t)^2 +(1-\delta) k_t + (1+r^{\star}+\pi(a_t))a_t\]
\[y_t = f(k_t, n_t, z_t)\]

\[z_{t+1} = \rho z_t + \epsilon_{t+1}\]

and \(u(c, n) = \frac{1}{1-\sigma}\left(c^{\psi}(1-n)^{1-\psi}
)\right)^{1-\sigma}\)
\end{frame}

\begin{frame}{Calibration}
\phantomsection\label{calibration}
\begin{columns}[T]
\begin{column}{0.48\textwidth}
\begin{longtable}[]{@{}ll@{}}
\toprule\noalign{}
Parameters & Values \\
\midrule\noalign{}
\endhead
\(σ\) & 2 \\
\(ψ\) & 1.45 \\
\(α\) & 0.32 \\
\(ω\) & 0.028 \\
\(r\) & 0.04 \\
\bottomrule\noalign{}
\end{longtable}
\end{column}

\begin{column}{0.48\textwidth}
\begin{longtable}[]{@{}ll@{}}
\toprule\noalign{}
Parameters & Values \\
\midrule\noalign{}
\endhead
\(δ\) & 0.1 \\
\(ρ\) & 0.42 \\
\(σ²\) & 0.0129 \\
\(A^{\star}\) & -0.7442 \\
\(χ\) & 0.000742 \\
\bottomrule\noalign{}
\end{longtable}
\end{column}
\end{columns}
\end{frame}

\begin{frame}{Results (1/2)}
\phantomsection\label{results-12}
\begin{figure}[H]

{\centering \includegraphics[width=\textwidth,height=0.6\textheight]{assets/irf.png}

}

\caption{Impulse Response Function}

\end{figure}%
\end{frame}

\begin{frame}{Results (2/2)}
\phantomsection\label{results-22}
\begin{figure}[H]

{\centering \includegraphics[width=\textwidth,height=0.6\textheight]{assets/sgu_moments.png}

}

\caption{Moments (from SGU)}

\end{figure}%
\end{frame}

\begin{frame}{Conclusions}
\phantomsection\label{conclusions}
\begin{itemize}
\tightlist
\item
  The model matches unconditional correlations fairly well

  \begin{itemize}
  \tightlist
  \item
    The stationarization device has little effect on the moments
  \end{itemize}
\item
  Unconditional correlations are not that great

  \begin{itemize}
  \tightlist
  \item
    a limitation of the moment matching method?
  \end{itemize}
\item
  Correlation of consumption with output is too high

  \begin{itemize}
  \tightlist
  \item
    what part of the model would you change?
  \end{itemize}
\end{itemize}
\end{frame}



\end{document}
