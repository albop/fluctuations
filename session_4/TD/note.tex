% Options for packages loaded elsewhere
\PassOptionsToPackage{unicode}{hyperref}
\PassOptionsToPackage{hyphens}{url}
\PassOptionsToPackage{dvipsnames,svgnames,x11names}{xcolor}
%
\documentclass[
  letterpaper,
  DIV=11,
  numbers=noendperiod]{scrartcl}

\usepackage{amsmath,amssymb}
\usepackage{lmodern}
\usepackage{iftex}
\ifPDFTeX
  \usepackage[T1]{fontenc}
  \usepackage[utf8]{inputenc}
  \usepackage{textcomp} % provide euro and other symbols
\else % if luatex or xetex
  \usepackage{unicode-math}
  \defaultfontfeatures{Scale=MatchLowercase}
  \defaultfontfeatures[\rmfamily]{Ligatures=TeX,Scale=1}
\fi
% Use upquote if available, for straight quotes in verbatim environments
\IfFileExists{upquote.sty}{\usepackage{upquote}}{}
\IfFileExists{microtype.sty}{% use microtype if available
  \usepackage[]{microtype}
  \UseMicrotypeSet[protrusion]{basicmath} % disable protrusion for tt fonts
}{}
\makeatletter
\@ifundefined{KOMAClassName}{% if non-KOMA class
  \IfFileExists{parskip.sty}{%
    \usepackage{parskip}
  }{% else
    \setlength{\parindent}{0pt}
    \setlength{\parskip}{6pt plus 2pt minus 1pt}}
}{% if KOMA class
  \KOMAoptions{parskip=half}}
\makeatother
\usepackage{xcolor}
\setlength{\emergencystretch}{3em} % prevent overfull lines
\setcounter{secnumdepth}{-\maxdimen} % remove section numbering
% Make \paragraph and \subparagraph free-standing
\ifx\paragraph\undefined\else
  \let\oldparagraph\paragraph
  \renewcommand{\paragraph}[1]{\oldparagraph{#1}\mbox{}}
\fi
\ifx\subparagraph\undefined\else
  \let\oldsubparagraph\subparagraph
  \renewcommand{\subparagraph}[1]{\oldsubparagraph{#1}\mbox{}}
\fi


\providecommand{\tightlist}{%
  \setlength{\itemsep}{0pt}\setlength{\parskip}{0pt}}\usepackage{longtable,booktabs,array}
\usepackage{calc} % for calculating minipage widths
% Correct order of tables after \paragraph or \subparagraph
\usepackage{etoolbox}
\makeatletter
\patchcmd\longtable{\par}{\if@noskipsec\mbox{}\fi\par}{}{}
\makeatother
% Allow footnotes in longtable head/foot
\IfFileExists{footnotehyper.sty}{\usepackage{footnotehyper}}{\usepackage{footnote}}
\makesavenoteenv{longtable}
\usepackage{graphicx}
\makeatletter
\def\maxwidth{\ifdim\Gin@nat@width>\linewidth\linewidth\else\Gin@nat@width\fi}
\def\maxheight{\ifdim\Gin@nat@height>\textheight\textheight\else\Gin@nat@height\fi}
\makeatother
% Scale images if necessary, so that they will not overflow the page
% margins by default, and it is still possible to overwrite the defaults
% using explicit options in \includegraphics[width, height, ...]{}
\setkeys{Gin}{width=\maxwidth,height=\maxheight,keepaspectratio}
% Set default figure placement to htbp
\makeatletter
\def\fps@figure{htbp}
\makeatother

\KOMAoption{captions}{tableheading}
\makeatletter
\makeatother
\makeatletter
\makeatother
\makeatletter
\@ifpackageloaded{caption}{}{\usepackage{caption}}
\AtBeginDocument{%
\ifdefined\contentsname
  \renewcommand*\contentsname{Table of contents}
\else
  \newcommand\contentsname{Table of contents}
\fi
\ifdefined\listfigurename
  \renewcommand*\listfigurename{List of Figures}
\else
  \newcommand\listfigurename{List of Figures}
\fi
\ifdefined\listtablename
  \renewcommand*\listtablename{List of Tables}
\else
  \newcommand\listtablename{List of Tables}
\fi
\ifdefined\figurename
  \renewcommand*\figurename{Figure}
\else
  \newcommand\figurename{Figure}
\fi
\ifdefined\tablename
  \renewcommand*\tablename{Table}
\else
  \newcommand\tablename{Table}
\fi
}
\@ifpackageloaded{float}{}{\usepackage{float}}
\floatstyle{ruled}
\@ifundefined{c@chapter}{\newfloat{codelisting}{h}{lop}}{\newfloat{codelisting}{h}{lop}[chapter]}
\floatname{codelisting}{Listing}
\newcommand*\listoflistings{\listof{codelisting}{List of Listings}}
\makeatother
\makeatletter
\@ifpackageloaded{caption}{}{\usepackage{caption}}
\@ifpackageloaded{subcaption}{}{\usepackage{subcaption}}
\makeatother
\makeatletter
\@ifpackageloaded{tcolorbox}{}{\usepackage[many]{tcolorbox}}
\makeatother
\makeatletter
\@ifundefined{shadecolor}{\definecolor{shadecolor}{rgb}{.97, .97, .97}}
\makeatother
\makeatletter
\makeatother
\ifLuaTeX
  \usepackage{selnolig}  % disable illegal ligatures
\fi
\IfFileExists{bookmark.sty}{\usepackage{bookmark}}{\usepackage{hyperref}}
\IfFileExists{xurl.sty}{\usepackage{xurl}}{} % add URL line breaks if available
\urlstyle{same} % disable monospaced font for URLs
\hypersetup{
  pdftitle={Climate Model and Perfect Foresight Simulation},
  colorlinks=true,
  linkcolor={blue},
  filecolor={Maroon},
  citecolor={Blue},
  urlcolor={Blue},
  pdfcreator={LaTeX via pandoc}}

\title{Climate Model and Perfect Foresight Simulation}
\author{}
\date{}

\begin{document}
\maketitle
\ifdefined\Shaded\renewenvironment{Shaded}{\begin{tcolorbox}[breakable, borderline west={3pt}{0pt}{shadecolor}, enhanced, boxrule=0pt, interior hidden, sharp corners, frame hidden]}{\end{tcolorbox}}\fi

Our ultimate goal consists in extending the neoclassical model to allow
for a (\emph{very}) reduced form of climate change, and understand the
qualitative implications. \footnote{in doing so, we follow broadly the
  approach from Integrated Assessment Models as the DICE model.}

Using Dynare timing conventions, our initial model corresponds to the
following equations (in \texttt{model\_1.mod}):

\begin{itemize}
\tightlist
\item
  capital accumlation: \[k_t = (1-\delta) k_{t-1} + i_t\]
\item
  budget constraint: \[c_t + i_t = y_t\]
\item
  production: \[y_t = exp(a_t)k_{t-1}^{\alpha}\]
\item
  marginal return on capital \[{mr}^k_t = \frac{y_t}{k_{t-1}}\]
\item
  optimality condition for consumption:
  \[\beta E_t \left[ \frac{U^{\prime}(c_{t+1})}{U^{\prime}(c_t)} (1-\delta + {mr}^k_t  )\right] = 1\]
\end{itemize}

Productivity shock \(a_t\) is exogenous.

\hypertarget{perfect-foresight-simulation-with-dynare}{%
\subsection{Perfect Foresight Simulation with
Dynare}\label{perfect-foresight-simulation-with-dynare}}

In the previous sessions we have used Dynare to compute stochastic
around a given steady-state. The shocks were i.i.d random variables.

This time, we will use the perfect foresight capabilities of Dynare to
compute a deterministic simulation between an initial and a final
steady-state. The driving exogenous process will be specified
deterministically and determine both the initial and the final
steady-states.

\begin{enumerate}
\def\labelenumi{\arabic{enumi}.}
\item
  \textbf{Check the \texttt{model\_1.mod} file. Modfify the initial
  steady-state so that the resid function returns zeros.}
\item
  \textbf{Uncomment the endval block and the resid function afterwards.
  This block sets the values of exogenous shock \texttt{a} in the final
  steady-state. Change the content of endval so that final steady-state
  is satisfied.}
\item
  \textbf{Uncomment the two perfect foresight instructions at the end
  and run the modfile. Use rplot to visualize the transition.
  Interpretation? What it is the value of \texttt{a} during the
  transition ?.}
\end{enumerate}

\emph{The simulation lasts for 50 periods. \texttt{a} jumps from 0 to
0.1 in period 1. During the transition, capital level is unchanged as
the final and the initial steady-states levels are the same.}

\begin{enumerate}
\def\labelenumi{\arabic{enumi}.}
\setcounter{enumi}{3}
\tightlist
\item
  \textbf{Add the following block before the endval command:}
\end{enumerate}

\begin{verbatim}
shocks;
var a;
periods 1:10;
values 0.0;
end;
\end{verbatim}

\begin{enumerate}
\def\labelenumi{\arabic{enumi}.}
\setcounter{enumi}{4}
\tightlist
\item
  \textbf{This command specifies that productivity shocks remains at
  zero during the first few periods, before jumping to its final value.
  Simulate, plot and comment on the new transition.}
\end{enumerate}

\emph{This time, we observe that in the first few periods, since
productivity is expected to rise in the future (perfect foresight),
current capital is comparatively less valueable. Agents consume it in
order to smooth consumption. Capital and investment decrease over time
while consumption has a growing path.}

\emph{Eventually the economy converges to the same steady-state as
before.}

\hypertarget{adding-an-energy-sector.}{%
\subsection{Adding an energy sector.}\label{adding-an-energy-sector.}}

We now modify the model so as to add a climate externality. To that
intent we create a new modfile \texttt{model\_2.mod}

We change the production function so as to add energy as a second input:

\[y_t = exp(a_t) k_{g,t-1}^{\alpha}e_t^{\alpha_e}\]

with \(\alpha_e=0.1\) and where \(k_{g,t}\) is the capital used to
produce final (consumption) goods.\footnote{Here the overall focus is on
  climate damages from energy consumption, which is why we don't add
  fuel stocks to the model.}

Energy itself is produced on competitive markets using capital
\(k_{e,t}\) according to production function:

\[e_{t} = k_{e,t-1}^{\alpha}\]

\begin{enumerate}
\def\labelenumi{\arabic{enumi}.}
\setcounter{enumi}{5}
\tightlist
\item
  \textbf{Which equation defines the market price of energy?}
\end{enumerate}

The price of energy is equal to its marginal productivity:
\[p_e = \alpha_e y_t / e_t\]

\begin{enumerate}
\def\labelenumi{\arabic{enumi}.}
\setcounter{enumi}{6}
\tightlist
\item
  \textbf{The resource constraint for capital is
  \(k_t = k_{e,t} + k_{g,t}\). What is now the marginal return of
  capital in both sectors ? What determines optimal allocation of
  capital between the consumption goods sector and the energy
  production?}
\end{enumerate}

The marginal return of capital is:

\begin{itemize}
\tightlist
\item
  \(\alpha y_t / k_{g,t-1}\) in the final goods sector
\item
  \(p_{e,t} = \alpha e_t / k_{e,t-1}\) in the energy sector
\end{itemize}

The fact that both are equal pins down the optimal allocation of capital
between both sectors.

\begin{enumerate}
\def\labelenumi{\arabic{enumi}.}
\setcounter{enumi}{7}
\tightlist
\item
  \textbf{Update the modile accordingly. Compute the new steady-state.}
\end{enumerate}

\emph{We can just use the steady-function: it should work given a good
enough initial guess for the capital.}

\hypertarget{damage-function}{%
\subsection{Damage function}\label{damage-function}}

We now add a schematic damage function in the spirit of DICE to the
model. We assume that the production of final goods is made according
to:

\[y_t = exp(a_t) (1-\omega_t) k_{g,t-1}^{\alpha}e_t^{\alpha_e}\]

where damage \(\omega_t\) is a damage that is a function of global
temperature \(T_t\):

\[\omega_t = {gw}_t \kappa_{\omega} T\]

with \(\kappa_{\omega} =0.1\). For practical reasons, we also add a
dummy variable \({gw}_t\) equal to 1, only if we take into account the
effects of global warming.

Global temperature itself cumulates the effects of past energy
consumption

\[T_t = \rho_T T_{t-1} +  \kappa_{e}  e_t\]

where autocorrelation term \(\rho_T=0.99\) is close to 1.

\begin{enumerate}
\def\labelenumi{\arabic{enumi}.}
\setcounter{enumi}{8}
\tightlist
\item
  \textbf{Add the new equations to the model. Which of the other
  equations need to be modified? Does the new setup has any effect on
  the price of energy?}
\end{enumerate}

Because of cobb-douglas production, the formula for the marginal return
on energy is unmodified. There is an effect on the price of energy
though, throught the adverse effect on productivity.

\begin{enumerate}
\def\labelenumi{\arabic{enumi}.}
\setcounter{enumi}{9}
\item
  \textbf{Make a perfect foresight simulation starting from an initial
  steady-state where \(gw_0=0\) and assuming
  \(\forall t\geq 1 gw_t = 1\). Comment on the transition and the new
  steady-state.}
\item
  \textbf{Assume the government levies a fixed, proportional tax on
  energy \(\tau>0\). Which equations do you need to change? Perform a
  deterministic simulation and compare to the former case.}
\end{enumerate}

\hypertarget{energy-transition}{%
\subsection{Energy transition}\label{energy-transition}}

Now we assume there are two energy sectors \(f\) and \(r\) (for fossil
and renewables). Capital is allocated between the two sectors in order
to produce:

\[e_{f,t} = k_{f,t-1}^{\alpha}\]
\[e_{r,t} = exp(\theta_r) k_{f,t-1}^{\alpha}\]

where exogenous shock \(\theta_r\) is the relative productivity of the
renewable sector with respect to the fossil one. For the sake of
simplicity, we assume that both kinds of energy are perfectly
substitutable to produce total energy \[e_t=e_{r,t}+e_{f,t}\].

We also make the assumption that global warming results from fossil
energy only:

\[T_t = \rho_T T_{t-1} +  \kappa_{e}  e_{f,t}\]

\begin{enumerate}
\def\labelenumi{\arabic{enumi}.}
\setcounter{enumi}{11}
\tightlist
\item
  \textbf{Create a new modfile model\_3, corresponding to the new model.
  What is the price of energy? Of fossil and renewable energy? What is
  the marginal return of capital in each sector?}
\end{enumerate}

The price of fossil and renewable energies are respectively:
\[p^e_{f,t} = \alpha_e y_t / e_t\] \[p^e_{r,t} = \alpha_e y_t / e_t\]

They are identical!

As for the marginal return of capital in each sector, it is

\[mr_{f,t} = \alpha y_t / k_{f,t-1}\]
\[mr_{f,t} = \alpha e_{f,t} / k_{f,t-1}\]
\[mr_{r,t} = \alpha e_{r,t} / k_{r,t-1}\]

The identity of the three options to invest capital defines optimal
allocation.

We observe that when \(\theta_t\) all formulas leading to capital
allocations are exactly the same for the two energy sectors.

\begin{enumerate}
\def\labelenumi{\arabic{enumi}.}
\setcounter{enumi}{12}
\tightlist
\item
  \textbf{We assume that \(\theta_{r,t}\) remains constant and equal to
  0. As in section 2, make a perfect foresight scenario with global
  warming.}
\end{enumerate}

Here we observe that despite, the negative externalities of fossil
fuels, the price of fossil fuels and the amount of capital allocated to
its production is exactly the same at the one allocated to renewable
energies.

\begin{enumerate}
\def\labelenumi{\arabic{enumi}.}
\setcounter{enumi}{13}
\tightlist
\item
  \textbf{What is the effect of adding a tax on fossil fuels? On the
  long run equilibrium? On the transition?}
\end{enumerate}

In this model, the effect of a proportional tax on fossil fuels that is
rebated to households as a lump-sum subsidy only shows up in the
definition of the price of fossil fuels:

\[p^e_{r,t} = \alpha_e y_t / e_t (1+\tau)\]

Setting up a higher \(\tau\) allows to avoid the climate damages, which
allows for higher consumption/production along the transition. The
effect is the same on the steady-state.

\begin{enumerate}
\def\labelenumi{\arabic{enumi}.}
\setcounter{enumi}{14}
\tightlist
\item
  \textbf{\emph{Propose} different ways to improve the model.}
\end{enumerate}



\end{document}
