% Options for packages loaded elsewhere
\PassOptionsToPackage{unicode}{hyperref}
\PassOptionsToPackage{hyphens}{url}
\PassOptionsToPackage{dvipsnames,svgnames,x11names}{xcolor}
%
\documentclass[
  letterpaper,
  DIV=11,
  numbers=noendperiod]{scrartcl}

\usepackage{amsmath,amssymb}
\usepackage{lmodern}
\usepackage{iftex}
\ifPDFTeX
  \usepackage[T1]{fontenc}
  \usepackage[utf8]{inputenc}
  \usepackage{textcomp} % provide euro and other symbols
\else % if luatex or xetex
  \usepackage{unicode-math}
  \defaultfontfeatures{Scale=MatchLowercase}
  \defaultfontfeatures[\rmfamily]{Ligatures=TeX,Scale=1}
\fi
% Use upquote if available, for straight quotes in verbatim environments
\IfFileExists{upquote.sty}{\usepackage{upquote}}{}
\IfFileExists{microtype.sty}{% use microtype if available
  \usepackage[]{microtype}
  \UseMicrotypeSet[protrusion]{basicmath} % disable protrusion for tt fonts
}{}
\makeatletter
\@ifundefined{KOMAClassName}{% if non-KOMA class
  \IfFileExists{parskip.sty}{%
    \usepackage{parskip}
  }{% else
    \setlength{\parindent}{0pt}
    \setlength{\parskip}{6pt plus 2pt minus 1pt}}
}{% if KOMA class
  \KOMAoptions{parskip=half}}
\makeatother
\usepackage{xcolor}
\setlength{\emergencystretch}{3em} % prevent overfull lines
\setcounter{secnumdepth}{-\maxdimen} % remove section numbering
% Make \paragraph and \subparagraph free-standing
\ifx\paragraph\undefined\else
  \let\oldparagraph\paragraph
  \renewcommand{\paragraph}[1]{\oldparagraph{#1}\mbox{}}
\fi
\ifx\subparagraph\undefined\else
  \let\oldsubparagraph\subparagraph
  \renewcommand{\subparagraph}[1]{\oldsubparagraph{#1}\mbox{}}
\fi


\providecommand{\tightlist}{%
  \setlength{\itemsep}{0pt}\setlength{\parskip}{0pt}}\usepackage{longtable,booktabs,array}
\usepackage{calc} % for calculating minipage widths
% Correct order of tables after \paragraph or \subparagraph
\usepackage{etoolbox}
\makeatletter
\patchcmd\longtable{\par}{\if@noskipsec\mbox{}\fi\par}{}{}
\makeatother
% Allow footnotes in longtable head/foot
\IfFileExists{footnotehyper.sty}{\usepackage{footnotehyper}}{\usepackage{footnote}}
\makesavenoteenv{longtable}
\usepackage{graphicx}
\makeatletter
\def\maxwidth{\ifdim\Gin@nat@width>\linewidth\linewidth\else\Gin@nat@width\fi}
\def\maxheight{\ifdim\Gin@nat@height>\textheight\textheight\else\Gin@nat@height\fi}
\makeatother
% Scale images if necessary, so that they will not overflow the page
% margins by default, and it is still possible to overwrite the defaults
% using explicit options in \includegraphics[width, height, ...]{}
\setkeys{Gin}{width=\maxwidth,height=\maxheight,keepaspectratio}
% Set default figure placement to htbp
\makeatletter
\def\fps@figure{htbp}
\makeatother

\KOMAoption{captions}{tableheading}
\makeatletter
\makeatother
\makeatletter
\makeatother
\makeatletter
\@ifpackageloaded{caption}{}{\usepackage{caption}}
\AtBeginDocument{%
\ifdefined\contentsname
  \renewcommand*\contentsname{Table of contents}
\else
  \newcommand\contentsname{Table of contents}
\fi
\ifdefined\listfigurename
  \renewcommand*\listfigurename{List of Figures}
\else
  \newcommand\listfigurename{List of Figures}
\fi
\ifdefined\listtablename
  \renewcommand*\listtablename{List of Tables}
\else
  \newcommand\listtablename{List of Tables}
\fi
\ifdefined\figurename
  \renewcommand*\figurename{Figure}
\else
  \newcommand\figurename{Figure}
\fi
\ifdefined\tablename
  \renewcommand*\tablename{Table}
\else
  \newcommand\tablename{Table}
\fi
}
\@ifpackageloaded{float}{}{\usepackage{float}}
\floatstyle{ruled}
\@ifundefined{c@chapter}{\newfloat{codelisting}{h}{lop}}{\newfloat{codelisting}{h}{lop}[chapter]}
\floatname{codelisting}{Listing}
\newcommand*\listoflistings{\listof{codelisting}{List of Listings}}
\makeatother
\makeatletter
\@ifpackageloaded{caption}{}{\usepackage{caption}}
\@ifpackageloaded{subcaption}{}{\usepackage{subcaption}}
\makeatother
\makeatletter
\@ifpackageloaded{tcolorbox}{}{\usepackage[many]{tcolorbox}}
\makeatother
\makeatletter
\@ifundefined{shadecolor}{\definecolor{shadecolor}{rgb}{.97, .97, .97}}
\makeatother
\makeatletter
\makeatother
\ifLuaTeX
  \usepackage{selnolig}  % disable illegal ligatures
\fi
\IfFileExists{bookmark.sty}{\usepackage{bookmark}}{\usepackage{hyperref}}
\IfFileExists{xurl.sty}{\usepackage{xurl}}{} % add URL line breaks if available
\urlstyle{same} % disable monospaced font for URLs
\hypersetup{
  pdftitle={A two agents model of inequalities.},
  colorlinks=true,
  linkcolor={blue},
  filecolor={Maroon},
  citecolor={Blue},
  urlcolor={Blue},
  pdfcreator={LaTeX via pandoc}}

\title{A two agents model of inequalities.}
\author{}
\date{}

\begin{document}
\maketitle
\ifdefined\Shaded\renewenvironment{Shaded}{\begin{tcolorbox}[enhanced, borderline west={3pt}{0pt}{shadecolor}, breakable, frame hidden, sharp corners, boxrule=0pt, interior hidden]}{\end{tcolorbox}}\fi

\hypertarget{preference-for-wealth-and-marginal-propensity-to-consume}{%
\subsection{Preference for wealth and marginal propensity to
consume}\label{preference-for-wealth-and-marginal-propensity-to-consume}}

For now, we consider a single representative agent. She has the ability
to buy a two periods bond, yielding 1 after one period. The price of the
bond at any date is \(q\), hence its (riskfree) interest rate is
\(r=1/q\).

Agent values consumption \(c_t\) and wealth \(b_t q_t\) so that she
maximizes\footnote{this is the ``preference for wealth'' specification}:

\[\max \sum_t \beta^t \left( \frac{c_t^{1-\frac{1}{\sigma}}}{1-\frac{1}{\sigma}}+ \varphi \frac{ (1+b_t)^{1-\frac{1}{\eta}} } {1-\frac{1}{\eta}} \right)\]

under the budget constraint

\[c_t = y_t + b_{t-1} - b_t q_t\]

where \(y_t\) is exogenous income following AR1

\[(y_t-\overline{y})=\rho (y_{t-1}-\overline{y}) + \epsilon^y_t\]

\textbf{1. Write down the optimality condition for debt holdings.}

\emph{Response:}

\[q_t = \beta \left(\frac{c_{t+1}}{c_{t}}\right)^{-\frac{1}{\sigma}}+\varphi \left(\frac{1 + b_t q_t}{c_{t}}\right)^{-\frac{1}{\sigma}}\]

\textbf{2. What are the equations defining the deterministic
equilibrium?}

\emph{Response:}
\[q = \beta + \varphi \left(\frac{1 + b q}{c}\right)^{-\frac{1}{\sigma}}\]

\textbf{3. Inspect \texttt{one\_agent.mod} model. Show that there is a
unit root. Can you interpret it?}

\emph{The \texttt{one\_agent.mod} model is a standard consumption saving
model without preference for wealth. Just add \texttt{check;} to see the
unit root. In this setup, any steady-state level of debt is feasible (in
a deterministic model).}

\textbf{4. What is the consumption response to a temporary income shock?
To a permanent one? (with autocorrelation \(\rho=0.9\) and
\(\rho=1.0\))}

\emph{Response:}

Change the income equation:

\begin{itemize}
\tightlist
\item
  for a temporary shock remove the zero to get:
  \texttt{y\ =\ ybar\ +\ e\_y}
\item
  For a persistent one, define a new parameter in the parameters line
  initialized to \texttt{rho=0.9} and change the equation to
  \texttt{y\ -\ ybar\ =\ rho*(y(-1)\ -\ ybar)\ +\ e\_y}. In the case
  \texttt{rho=1.0} note the apparition of a second unit root.
\end{itemize}

In the simulations, pay attention to the magnitude of the shock
\texttt{e\_y} (1\% by default) to compare it to the magnitude of the
response for assets \texttt{b}.

\textbf{5. In the modfile, add a preference for wealth term in the
utility function and adjust the calibration of \texttt{beta}
accordingly.}

\emph{Response}:

Parameters \texttt{phi} and \texttt{eta} are already predefined.

Add \texttt{+\ phi*(1+b*q)\^{}(-1/eta)/(c\^{}(-1/sigma))} to the Euler
equation and
\texttt{beta\ =\ 1/r\ -\ phi*(1+bbar*q)\^{}(-1/eta)/(cbar\^{}(-1/sigma));}
in the definition of parameters

Check the residuals (with command \texttt{resid;}) to ensure that
everything is fine. Incidentally we can see that one unit root has
disappeared, because pref. for wealth pins down equilibrium asset
holdings.

Result is in the \texttt{one\_agent\_2.mod} file.

\textbf{6. Simulate the response to a temporary and a persistent shock.
Given \texttt{phi} what is the effect of \texttt{eta}?}

\emph{Response}:

Now the savings response to transitory shock is mean reverting, while
the response to a persistent shock is persistent and even increasing
over time.

\texttt{eta} affects the long run savings level in response to a
temporary income shock. (If curious you should be able to check it does
not depend on phi).

\hypertarget{a-two-agents-model}{%
\subsection{A two agents model}\label{a-two-agents-model}}

We now assume there are two agents: bottom and top earners. Top earners
amount for a fraction \(\chi\) of the total population. Together they
earn a fraction \(z\in[0,1]\) of the total production \(y\) which is an
AR1 process as in the first part. The rest goes to the bottom earners.

Top earners can save by lending to bottom earners. We denote by \(B_t\)
the total quantity of riskfree bonds, traded at \(q_t\). Note that debt
per capita is \(\frac{B_t}{\chi}\) for top earners and
\(\frac{B_t}{1-\chi}\) for bottom earners. Top earners have preference
for wealth as in the first part, while bottom earners have standard
preferences (with \(\varphi=0\))

\textbf{1. Write down the budget equations for both agents. What are the
new Euler equations? Check that it is consistent with the
\texttt{two\_agents.mod} modfile. What are the per capita variables?}

\emph{Response}:

Budget constraints are included in the modfile.

Per capita variables: \texttt{c\_t}, \texttt{c\_b}, \texttt{b\_t} and
\texttt{b\_b}.

\textbf{2. What is qualitatively the effect of a permanent
redistributive shock? (simulate the model)}

\emph{Response}:

With increased income, top earners want to increase their asset
holdings. This will happen since bottom earners are indifferent in the
steady-state. During the transition however, we see a decrease in the
interest rate (to convince borrower to accept a decreasing path of
consumption over time).

\hypertarget{calibrating-and-simulating-the-model}{%
\subsection{Calibrating and simulating the
model}\label{calibrating-and-simulating-the-model}}

The model in the modefile is pre-calibrated to match US data in 1983.
Assume the model is in equilibrium for an initial level of debt
\(B=0.91\) (which is the debt/gdp ratio in the us in 1983).

Taking \(\varphi=0.05\) as constant, for any given choice of \(\eta\),
there is a unique value of \(\beta\) consistent with the equilibrium as
in the one agent case.

Now we would like to calibrate the value of \(\eta\) so as to match the
marginal propensity to save of top earners which was approximately 50\%
in 1983.

Since the two agents model is already calibrated for most variables, we
reuse it rather than adapting the one agent model.

\textbf{1. In the model \texttt{two\_agents.mod} replace the Euler
equation of bottom earner by \texttt{q\ =\ 1/rbar}. Justify why, from
the top earners perspective, the model is now equivalent to a single
agent model.}

\emph{Response}:

Interest rate is the only manifestation of bottom earners preferences in
the model. If we set it equal to a constant, top earners are now facing
an infinitely elastic demand of bonds as in the one agent case.

\textbf{2. Use the modified model to compute the marginal propensity out
of a permanent income shock after 6 periods. Choose parameter
\texttt{eta} so that this m.p.s. is approximatively 50\%.}

\emph{Response}:

Several options here. We can just make a persistent shock of size sig\_z
to the overall income \(y\) and compute the marginal propensity to save
at different periods (variable mps). It must be normalized by the
initial income level of top earners. We can then run computations for
different values of \(eta\) until we find the desired value.

Modified model is \texttt{two\_agents\_mps.mps}. I found eta=0.6 to be a
good calibration.

\textbf{3. What is the effect of a 10\% permanent increase in
inequalities? Over 30 years and in the long run?}

\emph{Response}:

We can set the value of eta that we found, increase the standard
deviation of sig\_z to 0.1 and change the simulation horizon (first to
30 then to 200).

Over 30 years I found an increase of 25 of the debt/gdp level
(\texttt{B}). Over 200 years it amounts to 0.6/0.7.



\end{document}
